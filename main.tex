%% Copyright 2006-2013 Xavier Danaux (xdanaux@gmail.com).
%
% This work may be distributed and/or modified under the
% conditions of the LaTeX Project Public License version 1.3c,
% available at http://www.latex-project.org/lppl/.

% basic configurations of the file, change ever i want
\documentclass[12pt,a4paper,roman]{moderncv} 
\moderncvstyle{classic}                             % style options are 'casual' (default), 'classic', 'oldstyle' and 'banking'
\moderncvcolor{black}                               % color options 'blue' (default), 'orange', 'green', 'red', 'purple', 'grey' and 'black'

\usepackage[utf8]{inputenc}

% adjust the page margins
\usepackage[scale=0.8]{geometry}
%\setlength{\hintscolumnwidth}{3cm}                % if you want to change the width of the column with the dates
%\setlength{\makecvtitlenamewidth}{10cm}           % for the 'classic' style, if you want to force the width allocated to your name and avoid line breaks. be careful though, the length is normally calculated to avoid any overlap with your personal info; use this at your own typographical risks...

% personal data
\name{Rodrigo}{Lopes Rincon}
\title{Engenheiro de Software residente em Brasília}

\phone[mobile]{(61)99954-6672}
\phone[fixed]{(61)3264-7434}
\email{rodrigolr\_15@hotmail.com}
\homepage{github.com/rodrigorincon}
%\extrainfo{aniversário: 29/11/1991}
%\photo[64pt][0.4pt]{picture}


\quote{Sempre atento com qualidade de código e às boas práticas. Apreciador de desenvolvimento ágil e de um bom achocolatado.}

\begin{document}

\makecvtitle

\section{Formação Acadêmica}
\cventry{2009--2015}{Bacharelado}{}{Engenharia de Software pela Universidade de Brasília}{}{}  % arguments 3 to 6 can be left empty
\cventry{2010--2013}{Técnico}{}{curso de Computação Gráfica "Start" pela School of Art, Game and Animation (SAGA)}{}{}

\section{Experiências}
\cventry{Setembro 2013--Julho 2014}{Estágio}{}{Banco do Brasil}{}{Manutenção e
evolução de sistemas web em Java usados pelas agências fora do país.\newline{}
Tarefas realizadas:
\begin{itemize}
    \item Refatoração e evolução de sistema Java web;
      \begin{itemize}
          \item Modularização do sistema;
          \item Limpeza do código de bad smells;
          \item Desenvolvimento de novas funcionalidades;
      \end{itemize}
    \item Manutenção na base de dados em PostgreSQL;
    \item Iniciado o desenvolvimento de uma API RESTFull Java para acesso pelas filiais à dados da base da sede nacional;
    \item Frameworks e tecnologias utilizadas:
      \begin{itemize}
          \item JSF;
          \item framework FreeMarker;
          \item PostgreSQL;
          \item SVN;
      \end{itemize}
    \item Contato: Carlos (61)98477-2967.
\end{itemize}}

\cventry{Julho 2015--Dezembro 2015}{Co-fundador}{}{Coremaster}{}{Co-fundador da empresa juntamente a um amigo, na categoria Simples Nacional. Empresa atuava no desenvolvimento de sites institucionais e estruturação e segurança de redes de computadores. Saí da empresa devido a divergências sobre a visão de negócio com o fundador.\newline{}
Tarefas realizadas:
\begin{itemize}
    \item Construção sites em HTML5 e CSS3;
    \item Planejamento das estratégias de marketing digital da empresa;
    \item Gestão do fluxo de caixa da empresa;
    \item Frameworks e tecnologias utilizadas:
      \begin{itemize}
          \item Jquery;
          \item Bootstrap;
          \item Wordpress;
      \end{itemize}
    \item Contato: Rafael Cançado (61)98102-8023.
\end{itemize}}

\cventry{Abril 2016--Dezembro 2016}{Desenvolvedor Ruby on Rails e Nodejs}{}{Preditiva}{}{Desenvolvedor web na startup, implementando novas funcionalidades, soluções para o processo de deploy e comunicação com os hardwares instalados no cliente.\newline{}
Tarefas realizadas:
\begin{itemize}
    \item Novas funcionalidades no sistema web;
    \item Correções e melhorias no MVP para torná-lo um produto mais robusto;
    \item Soluções para integração contínua;
    \item Comunicação real-time com os hardwares instalados nos clientes usando Nodejs;
    \item Criação de diversas landing pages para validação de ideias com HTML5 e CSS3;
    \item Integração das landing pages com as ferramenta Google Analytics e Google Ads;
    \item Frameworks e tecnologias utilizadas:
      \begin{itemize}
          \item Ruby on Rails;
          \item Scrum;
          \item JQuery;
          \item Bootstrap;
          \item Nodejs;
          \item SemaphoreCI;
          \item Git;
          \item Google Analytics e Google Ads;
      \end{itemize}
    \item Contato: Luiz (61)99203-3355.
\end{itemize}}

\cventry{Dezembro 2016--Hoje}{Desenvolvedor Ruby on Rails}{}{AIS Digital}{}{Desenvolvendo e dando manutenção em sistemas para grandes clientes, entre eles Banco do Brasil e Altran. Participação em projetos internacionais, desde a comunicação direta com o cliente, levantamento das necessidades com o mesmo e resolução deles. \newline{}
Tarefas realizadas:
\begin{itemize}
    \item Desenvolvimento de sistemas do início e manutenção em sistemas legados em Ruby on Rails;
    \item Estimar e projetar esforço e impacto de novas implementações nos projetos;
    \item Implementação do Scrum e Kanban nas equipes as quais participei;
    \item Desenvolvimento de APIs para comunicação com frontend e aplicativos mobile;
    \item Implantação de soluções para integração contínua, análise de código e monitoramento de falhas;
    \item Comunicação direta com cliente: atendendo a chamadas, sendo alocado no mesmo e realizando levantamento de necessidades e bugs com cliente nacional e internacional;
    \item Documentação do projeto, API e processos;
    \item Frameworks e tecnologias utilizadas:
      \begin{itemize}
          \item Ruby on Rails e Capistrano;
          \item Scrum e Kanban;
          \item JQuery;
          \item Backbone JS;
          \item Vuejs;
          \item Docker;
          \item Git e SVN;
          \item New Relic e Sentry;
          \item SonarQube e Rubocop;
          \item Swagger, Wiki e Confluence;
          \item AWS e Jelastic;
      \end{itemize}
\end{itemize}}



\section{Cursos de Capacitação à distância}
\begin{itemize}
    \item Fundamentos de Rede - CIEE (2008), Fundação Bradesco (2012)
    \item Segurança da Informação - Fundação Bradesco (2012)
    \item Análise Estruturada de Sistemas - Fundação Bradesco (2012)
    \item Modelagem de Dados - Fundação Bradesco (2012)
    \item Processo de Desenvolvimento de Software - Fundação Bradesco (2012)
    \item Engenharia de Requisitos - Fundação Bradesco (2012)
    \item Vue.js Essentials - Udemy (2017)
\end{itemize}

\section{Cursos de Capacitação presenciais}
\begin{itemize}
    \item Linux Básico - Universidade de Brasília – UnB (2009)
    \item HTML para engenharia - Universidade de Brasília – UnB (2010)
    \item Adobe Photoshop CS4 - SAGA (2010)
    \item Adobe Illustrator CS5 - SAGA (2010)
    \item Adobe Dreamweaver CS5 - SAGA (2011)
\end{itemize}
\section{Idiomas}
\cvitemwithcomment{Portugês}{Nativo}{}
\cvitemwithcomment{Inglês}{Intermediário}{}

\section{Conhecimentos Técnicos}
{\fontsize{20}{30}\selectfont \textbf{Especialidades}}\newline{}

\cvitem{Linguagens}{Java, Javascript, Ruby}
\cvitem{Tecnologias web}{HTML, JQuery, Bootstrap, arquiteturas REST, Ruby on Rails}
\cvitem{Banco de dados SQL}{MySQL, PostgreSQL, Oracle}
\cvitem{Metodologias e práticas}{Scrum, Kanban, XP}
\cvitem{Ferramentas de apoio ao desenvolvimento}{Git, UML, Docker}
\cvitem{Ferramentas de testes}{Junit, Rspec, Cucumber, Jasmine}
\cvitem{Sistemas operacionais}{Debian, Windows}

{\fontsize{20}{30}\selectfont \textbf{Conheço bem}}\newline{}

\cvitem{Linguagens}{Python, C, C++}
\cvitem{Tecnologias web}{Nodejs, Backbonejs, Vuejs, Apache, Passenger}
\cvitem{Ferramentas de apoio ao desenvolvimento}{Rubocop, Sentry, Pipeline}
\cvitem{Áreas de conhecimento}{Arquitetura de software, Desenvolvimento de frameworks, Padrões de projeto, Certificados e Assinaturas digitais, Criptografia simétrica e assimétrica}

{\fontsize{20}{30}\selectfont \textbf{Conhecimento básico}}\newline{}

\cvitem{Metodologias e práticas}{RUP, PMBOK}
\cvitem{Tecnologias web}{CSS, PHP, Java web, AWS}
\cvitem{Ferramentas de apoio ao desenvolvimento}{SVN, Capistrano, SonarQube, Vagrant, Swagger}
\cvitem{Áreas de conhecimento: processos}{Medição e estimativa de custos e prazos (incluindo análise de pontos de função), Elicitação de requisitos, Melhoria, Modelagem e Implantação de processo de software} 
\cvitem{Áreas de conhecimento: desenvolvimento}{OpenSSL, robótica}
%\section{Interesses}
\cvitem{Robótica}{Um dos assuntos que mais me atraem a atenção e gosto de estudar sobre quando tenho tempo sobrando.}
\cvitem{Astronomia}{Área que desde pequeno me encanta, estou sempre lendo sobre e acompanhando as novidades da exploração espacial e suas descobertas.}
\cvitem{Data Science}{Uma das áreas que pretendo estudar no futuro próximo e compreender como funciona os algoritmos usados por este setor da TI.}
\section{Apresentações e projetos}

\begin{itemize}
    \item Apresentei-me no Encontro de Ciência e Tecnologia da UnB – Gama (ECT 2011) com o projeto de software de reeducação alimentar para sistema operacional Android com apresentação de protótipo;
    \item Apresentei oralmente e tive artigo aceito nos anais do Encontro de Ciência e Tecnologia da UnB – Gama (ECT 2011) com o projeto de capacitação continuada e tecnologias apropriadas para inclusão digital de deficientes físicos e idosos na Adapte;
    \item Participei do projeto de pesquisa Casa Brasil da Adapte (Associação de Apoio aos Portadores de Necessidades Especiais) pela Universidade de Brasília, no ano de 2011, na área de integração de deficientes físicos e idosos à informática;
    \item Participei de projeto de desenvolvimento na Universidade de Brasília, no ano de 2012, na área de banco de dados e desenvolvimento de projetos em Java;
    \item Participei do projeto Carona Solidária na Universidade de Brasília, no ano de 2013 à 2014, como programador Java e PHP e responsável pela modelagem e implementação do banco de dados;
    \item Participei do desenvolvimento do plugin para o Adobe Reader para validação de assinaturas digitais no padrão PAdES da ICP-Brasil pela Universidade de Brasília, nos anos de 2015 e 2016, utilizando OpenSSL e a linguagem C++;
    \item Participei como ouvinte dos congressos e eventos de tecnologia: Campus Party (2014, 2015 e 2016), Campus Party Brasília (2017) e Latinoware (2014).
\end{itemize}

%%%%%%%%%%%%%%%%%%%%%%%%%%%%%%%%%%%%%%%%%%%%%%%%%%%%%%%%%%%%%%%%%%%%%%%%%%%
% OUTRAS FORMAS DE FAZER UMA SESSÃO COM DADOS COM <ul>
%%%%%%%%%%%%%%%%%%%%%%%%%%%%%%%%%%%%%%%%%%%%%%%%%%%%%%%%%%%%%%%%%%%%%%%%%%%
%\section{Extra 1}
%\cvlistitem{Item 1}
%\cvlistitem{Item 2}
%\cvlistitem{Item 3. This item is particularly long and therefore normally spans over several lines. Did you notice the indentation when the line wraps?}

%\section{Extra 2}
%\cvlistdoubleitem{Item 1}{Item 4}
%\cvlistdoubleitem{Item 2}{Item 5\cite{book1}}
%\cvlistdoubleitem{Item 3}{Item 6. Like item 3 in the single column list before, this item is particularly long to wrap over several lines.}

%\section{References}
%\begin{cvcolumns}
%  \cvcolumn{Category 1}{\begin{itemize}\item Person 1\item Person 2\item Person 3\end{itemize}}
%  \cvcolumn{Category 2}{Amongst others:\begin{itemize}\item Person 1, and\item Person 2\end{itemize}(more upon request)}
%  \cvcolumn[0.5]{All the rest \& some more}{\textit{That} person, and \textbf{those} also (all available upon request).}
%\end{cvcolumns}

% Publications from a BibTeX file without multibib
%  for numerical labels: \renewcommand{\bibliographyitemlabel}{\@biblabel{\arabic{enumiv}}}% CONSIDER MERGING WITH PREAMBLE PART
%  to redefine the heading string ("Publications"): \renewcommand{\refname}{Articles}
%\nocite{*}
%\bibliographystyle{plain}
%\bibliography{publications}                        % 'publications' is the name of a BibTeX file

% Publications from a BibTeX file using the multibib package
%\section{Publications}
%\nocitebook{book1,book2}
%\bibliographystylebook{plain}
%\bibliographybook{publications}                   % 'publications' is the name of a BibTeX file
%\nocitemisc{misc1,misc2,misc3}
%\bibliographystylemisc{plain}
%\bibliographymisc{publications}                   % 'publications' is the name of a BibTeX file

\end{document}
%% end of file `template.tex'.
