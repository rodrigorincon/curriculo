\section{Experiências}
\cventry{Setembro 2013--Julho 2014}{Estágio}{}{Banco do Brasil}{}{Manutenção e
evolução de sistemas web em Java usados pelas agências fora do país.\newline{}
Tarefas realizadas:
\begin{itemize}
    \item Refatoração e evolução de sistema Java web;
      \begin{itemize}
          \item Modularização do sistema;
          \item Limpeza do código de bad smells;
          \item Desenvolvimento de novas funcionalidades;
      \end{itemize}
    \item Manutenção na base de dados em PostgreSQL;
    \item Iniciado o desenvolvimento de uma API RESTFull Java para acesso pelas filiais à dados da base da sede nacional;
    \item Frameworks e tecnologias utilizadas:
      \begin{itemize}
          \item JSF;
          \item framework FreeMarker;
          \item PostgreSQL;
          \item SVN;
      \end{itemize}
    \item Contato: Carlos (61)98477-2967.
\end{itemize}}

\cventry{Julho 2015--Dezembro 2015}{Co-fundador}{}{Coremaster}{}{Co-fundador da empresa juntamente a um amigo, na categoria Simples Nacional. Empresa atuava no desenvolvimento de sites institucionais e estruturação e segurança de redes de computadores. Saí da empresa devido a divergências sobre a visão de negócio com o fundador.\newline{}
Tarefas realizadas:
\begin{itemize}
    \item Construção sites em HTML5 e CSS3;
    \item Planejamento das estratégias de marketing digital da empresa;
    \item Gestão do fluxo de caixa da empresa;
    \item Frameworks e tecnologias utilizadas:
      \begin{itemize}
          \item Jquery;
          \item Bootstrap;
          \item Wordpress;
      \end{itemize}
    \item Contato: Rafael Cançado (61)98102-8023.
\end{itemize}}

\cventry{Abril 2016--Dezembro 2016}{Desenvolvedor Ruby on Rails e Nodejs}{}{Preditiva}{}{Desenvolvedor web na startup, implementando novas funcionalidades, soluções para o processo de deploy e comunicação com os hardwares instalados no cliente.\newline{}
Tarefas realizadas:
\begin{itemize}
    \item Novas funcionalidades no sistema web;
    \item Correções e melhorias no MVP para torná-lo um produto mais robusto;
    \item Soluções para integração contínua;
    \item Comunicação real-time com os hardwares instalados nos clientes usando Nodejs;
    \item Criação de diversas landing pages para validação de ideias com HTML5 e CSS3;
    \item Integração das landing pages com as ferramenta Google Analytics e Google Ads;
    \item Frameworks e tecnologias utilizadas:
      \begin{itemize}
          \item Ruby on Rails;
          \item Scrum;
          \item JQuery;
          \item Bootstrap;
          \item Nodejs;
          \item SemaphoreCI;
          \item MySQL;
          \item Git;
          \item Google Analytics e Google Ads;
      \end{itemize}
    \item Contato: Luiz (61)99203-3355.
\end{itemize}}

\cventry{Dezembro 2016--Hoje}{Desenvolvedor Ruby on Rails}{}{AIS Digital}{}{Desenvolvendo e dando manutenção em sistemas para grandes clientes, entre eles Banco do Brasil e Altran. Participação em projetos internacionais, desde a comunicação direta com o cliente, levantamento das necessidades com o mesmo e resolução deles. \newline{}
Tarefas realizadas:
\begin{itemize}
    \item Desenvolvimento de sistemas do início e manutenção em sistemas legados em Ruby on Rails;
    \item Estimar e projetar esforço e impacto de novas implementações nos projetos;
    \item Implementação do Scrum e Kanban nas equipes as quais participei;
    \item Desenvolvimento de APIs para comunicação com frontend e aplicativos mobile;
    \item Implantação de soluções para integração e deploy contínuo, análise de código e monitoramento de falhas;
    \item Comunicação direta com cliente: atendendo a chamadas, sendo alocado no mesmo e realizando levantamento de necessidades e bugs com cliente nacional e internacional;
    \item Documentação do projeto, API e processos;
    \item Frameworks e tecnologias utilizadas:
      \begin{itemize}
          \item Ruby on Rails e Capistrano;
          \item Scrum e Kanban;
          \item JQuery;
          \item Backbone JS;
          \item Vuejs;
          \item MySQL e PostgreSQL;
          \item Docker;
          \item Git e SVN;
          \item Pipeline;
          \item New Relic e Sentry;
          \item SonarQube e Rubocop;
          \item Swagger, Wiki e Confluence;
          \item AWS e Jelastic;
      \end{itemize}
\end{itemize}}

