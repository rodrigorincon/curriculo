\section{Apresentações}

\begin{itemize}
    \item Dei um workshop sobre algoritmos e como pensar soluções como um programador na Feira Latina Americana de Instalação de Software Livre (FLISoL 2023), o maior evento de divulgação de software livre na América Latina, e na Campus Party 2023, focado para crianças no ensino fundamental;
    \item Dei um workshop sobre primeiros passos em programação com Javascript na Campus Party 2022, ensinando os conceitos básico sobre programação usando Javascript para adolescentes e adultos;
    \item Dei alguns workshops e palestras na empresa Epy para os estagiários durante 2021 and 2022, ensinando sobre boas práticas, clean code, integrações com API e componentes específicos do sistema interno;
    \item Dei um workshop e palestra sobre testes automatizados com RSpec na empresa AIS Digital em 2018, falando sobre o que e como testar, os tipos de testes e como escrever testes com RSpec em projetos em Ruby on Rails;
    \item Participei do projeto Edubot na Universidade de no ano de 2016, ensinando conceitos básicos de programação para alunos do ensino médio através da robótica, utilizando um kit educacional feito a partir do Arduíno;
    \item Apresentei-me no Encontro de Ciência e Tecnologia da UnB – Gama (ECT 2011) com o projeto de software de reeducação alimentar para sistema operacional Android com apresentação de protótipo;
    \item Apresentei oralmente e tive artigo aceito nos anais do Encontro de Ciência e Tecnologia da UnB – Gama (ECT 2011) com o projeto de capacitação continuada e tecnologias apropriadas para inclusão digital de deficientes físicos e idosos na Adapte;
\end{itemize}


\section{Projetos}

\begin{itemize}
    \item Participei da ONG Labkids durante 2022 e 2023, ensinando conceitos básicos de programação para crianças através de robótica, usando kits educacionais feitos a partir do Arduíno;
    \item Participei do projeto Edubot na Universidade de Brasília no ano de 2016, ensinando conceitos básicos de programação para alunos do ensino médio através da robótica, utilizando um kit educacional feito a partir do Arduíno;
    \item Participei do desenvolvimento do plugin para o Adobe Reader para validação de assinaturas digitais no padrão PAdES da ICP-Brasil pela Universidade de Brasília, nos anos de 2015 e 2016, utilizando OpenSSL e a linguagem C++;
    \item Participei do projeto Carona Solidária na Universidade de Brasília, no ano de 2013 à 2014, como programador Java e PHP e responsável pela modelagem e implementação do banco de dados;
    \item Participei de projeto de desenvolvimento na Universidade de Brasília, no ano de 2012, na área de banco de dados e desenvolvimento de projetos em Java;
    \item Participei do projeto de pesquisa Casa Brasil da Adapte (Associação de Apoio aos Portadores de Necessidades Especiais) pela Universidade de Brasília, no ano de 2011, na área de integração de deficientes físicos e idosos à informática;
\end{itemize}