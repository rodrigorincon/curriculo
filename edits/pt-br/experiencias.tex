\section{Experiências}

\cventry{Janeiro 2024--Hoje}{CTO}{}{Oto.Live}{}{Trabalho junto ao CEO para definir como será o sistema de gamificação de eventos e transformo as decisões em regras de negócios, requisitos e alinho as necessidades com o time técnico. Criei a POC e gerencio o desenvolvimento das novas versões. Sou responsável por definir requisitos funcionais e não funcionais, infraestrutura, escalabilidade e trabalhar com as equipes de desenvolvimento, QA e design, escrevendo a documentação para eles e desenvolvendo pessoalmente partes cruciais do backend. Como a equipe técnica está espalhada em diferentes continentes, sou responsável por gerenciá-los e manter toda a documentação atualizada para que a equipe possa trabalhar de forma assíncrona.\newline{}
Tarefas realizadas:
\begin{itemize}
    \item Definir regras de negócio e os requisitos;
    \item Atuar como ponte entre a equipe técnica e o CEO e demais diretores;
    \item Escrever documentação;
    \item Projetar o sistema e a infraestrutura;
    \item Gerenciar uma equipe multidisciplinar global com toda a comunicação em inglês;
    \item Definir soluções de segurança e escalabilidade;
    \item Desenvolver POC do zero;
    \item Atuo como gerente de operação durante grandes eventos;
    \item Frameworks e tecnologias utilizadas:
      \begin{itemize}
          \item Ruby on Rails;
          \item RSpec e Rubocop;
          \item AWS;
          \item AWS Rekognition e ChatGPT API;
          \item Microsserviços com AWS Lambda;
          \item Typescript e NextJs;
          \item Kanbam;
          \item PostgreSQL e Redis;
          \item Docker, Git e Swagger;
      \end{itemize}
\end{itemize}}


\cventry{Setembro 2022--Julho 2025}{Desenvolvedor Backend}{}{GoDaddy}{}{Trabalho em uma empresa dos Estados Unidos em uma equipe espalhada por todo o globo, com comunicação toda feita em inglês. Eu trabalho próximo e fazendo reuniões diárias com membros em fuso horários diferentes. Eu construo novas funcionalidades e dando manutenção no sistema interno, especialmente integração com marketplaces externos (como Amazon e Ebay) e garantindo que qualquer atualização em qualquer marketplace será refletido nos demais. \newline{}
Tarefas realizadas:
\begin{itemize}
    \item Desenvolvimento de novas funcionalidades;
    \item Projeto e implementação de novos componentes;
    \item Correção de bugs e investigação de problemas reportados pelo cliente;
    \item Participar de plantões periódicos com o time de suporte;
    \item Documentação de migrações importante, integrações e políticas de seguança;
    \item Comunicação com APIs externas, ferramentas e frontend;
    \item Escrita de testes automatizados com RSpec;
    \item Atuar em uma equipe multidisciplinar espalhado pelo globo com toda a comunicação em inglês;
    \item Frameworks e tecnologias utilizadas:
      \begin{itemize}
          \item Ruby on Rails;
          \item RSpec;
          \item AWS;
          \item Typescript;
          \item Microsservices com AWS Lambda;
          \item Scrum;
          \item PostgreSQL, Redis e DynamoDB;
          \item Docker;
          \item Git;
          \item Swagger, Confluence e Github Pages;
          \item Honeybadger e Mixpanel;
          \item Rubocop;
      \end{itemize}
\end{itemize}}

\cventry{Fevereiro 2022--Setembro 2022}{Desenvolvedor Backend}{}{Sólides}{}{Trabalhei construindo novas soluções e na manutenção do sistema de gestão de vagas de emprego. Desenvolvi novas funcionalidades na nova versão da API e migrei componentes da versão antiga para a nova. A maior parte do tempo foi estive na equipe de manutenção do código, trabalhando junto com o time de suporte. \newline{}
Tarefas realizadas:
\begin{itemize}
    \item Correção de bugs e investigação de problemas com o time de suporte;
    \item Desenvolvimento de novas funcionalidades em Ruby on Rails;
    \item Migração das funcionalidades atuais para a nova versão da API;
    \item Escrever documentação sobre os componentes trabalhados;
    \item Comunicação entre a API Ruby e o frontend React;
    \item Frameworks e tecnologias utilizadas:
      \begin{itemize}
          \item Ruby on Rails;
          \item React;
          \item Kanban;
          \item PostgreSQL;
          \item Docker;
          \item Git;
          \item Swagger, Wiki e Confluence;
          \item Rubocop;
      \end{itemize}
\end{itemize}}

\cventry{Julho 2020--Março 2022}{Desenvolvedor FullStack}{}{Epy}{}{Desenvolvi o sistema de atendimento, delivery e pagamento para restaurantes, permitindo gerenciar pedidos e pagamentos. Também construí o sistema mobile PWA para usuário acessarem o menu, pedidos e pagarem pelo celular e acompanharem as entregas. Atuei em todas as fases do desenvolvimento, desde a concepção da nova funcionalidade, arquitetura e implementação. Atuei também no monitoramento e otimização do sistema de forma constante. Realizei integrações com diversos sistemas externos. Migrei o sistema em Ruby on Rails para uma arquitetura totalmente nova em microsserviços. Também fui responsável por ser mentor dos trainees da startup e ajudá-los nos seus desenvolvimentos. \newline{}
Tarefas realizadas:
\begin{itemize}
    \item Desenvolvimento de componentes e funcionalidades em Ruby on Rails e React;
    \item Implementação do sistema PWA no celular e comunicação em tempo real entre todas as partes (celular, servidor e desktop);
    \item Acompanhamento e mentoria dos trainees;
    \item Dei workshops e palestras para os trainees;
    \item Otimização de pesquisas no banco de dados;
    \item Refatorações no sistema para torná-lo mais rápido;
    \item Autenticação e integração com APIs externas;
    \item Modelagem e implementação de uma nova arquitetura em microsserviços usando lambdas da AWS;
    \item Frameworks e tecnologias utilizadas:
      \begin{itemize}
          \item Ruby on Rails;
          \item Javascript e Typescript;
          \item React;
          \item PWA;
          \item Web sockets;
          \item AWS;
          \item Microsserviços;
          \item Scrum;
          \item PostgreSQL;
          \item Docker;
          \item Git;
          \item Sentry;
          \item Rubocop;
      \end{itemize}
\end{itemize}}

\cventry{Abril 2019--Novembro 2019}{Desenvolvedor FullStack}{}{Ownee}{}{Desenvolvi sistema de gerenciamento de obras para construção civil. Construí novas funcionalidades, otimizei as requisições e as buscas no banco de dados, comuniquei com APIs externas e criei componentes novos tanto no backend como no frontend. Foi feito uma refatoração completa, alterando o núcleo do sistema, a arquitetura e suas regras de negócio, aonde atuei ativamente na modelagem da nova versão. Atuei como analista de testes, sendo o responsável por escrevê-los e atualizá-los a cada mudança nos requisitos. Atuei como líder técnico ao lado do CTO e guiei os estagiários em suas atividades. \newline{}
Tarefas realizadas:
\begin{itemize}
    \item Modelagem e implementação de uma nova arquitetura a partir das mudanças no sistema e da versão antiga;
    \item Desenvolvimento de componentes e funcionalidades em Ruby on Rails e Vuejs;
    \item Acompanhamento e mentoria dos estagiários;
    \item Otimização de pesquisas no banco de dados;
    \item Refatorações no sistema para torná-lo mais rápido e escalável;
    \item Investigação e correção de bugs reportados pelo usuários;
    \item Comunicação entre frontend e backend e com APIs externas;
    \item Manutenção dos testes e criação dos novos;
    \item Frameworks e tecnologias utilizadas:
      \begin{itemize}
          \item Ruby on Rails e RSpec;
          \item Scrum;
          \item Javascript
          \item Vuejs;
          \item PostgreSQL;
          \item Docker;
          \item Git;
          \item New Relic;
          \item Wiki e AWS;
      \end{itemize}
\end{itemize}}

\cventry{Dezembro 2016--Abril 2019}{Desenvolvedor FullStack}{}{AIS Digital}{}{Desenvolvimento e manutenção em sistemas Ruby on Rails para grandes clientes, entre eles Banco do Brasil e Altran. Participação em projetos internacionais, desde a comunicação direta com o cliente, levantamento das necessidades com o mesmo e resolução deles. Atuei em equipes multi-disciplinares e dinâmicas, desenvolvendo o frontend ao lado dos designers e contruindo APIs para o frontend e os apps mobile acessarem. Palestrei e realizei workshop para a equipe de desenvolvimento da empresa sobre testes automatizados. \newline{}
Tarefas realizadas:
\begin{itemize}
    \item Desenvolvimento de sistemas do início e manutenção em sistemas legados em Ruby on Rails;
    \item Realizar palestra e workshop sobre testes automatizados;
    \item Atuação em equipes multi-disciplinares em formato de squads;
    \item Desenvolvimento de APIs para comunicação com frontend e aplicativos mobile;
    \item Implantação de soluções para integração e deploy contínuo, análise de código e monitoramento de falhas;
    \item Comunicação direta com cliente: atendendo a chamadas, sendo alocado no mesmo e realizando levantamento de necessidades e bugs com cliente nacional e internacional;
    \item Documentação do projeto, API e processos;
    \item Frameworks e tecnologias utilizadas:
      \begin{itemize}
          \item Ruby on Rails;
          \item NodeJS e Javascript;
          \item Scrum e Kanban;
          \item JQuery e Backbone JS;
          \item Vuejs;
          \item RSpec, Capistrano e Jest;
          \item MySQL e PostgreSQL;
          \item Docker;
          \item Git e SVN;
          \item Bitbucket Pipeline;
          \item New Relic e Sentry;
          \item SonarQube e Rubocop;
          \item Swagger, Wiki e Confluence;
          \item AWS e Jelastic;
      \end{itemize}
\end{itemize}}

\cventry{Abril 2016--Dezembro 2016}{Desenvolvedor Ruby on Rails e Nodejs}{}{Preditiva}{}{Trabalhei no sistema de monitoramento de temperatura e umidade da startup. Atuei corrigindo bugs e implementando novas funcionalidades. Implantei soluções de deploy contínuo e construí diversas landing pages para validação de ideias pela equipe de marketing. Construí ainda um middleware para facilitar a comunicação em tempo real com os hardwares instalados no cliente e permitir a comunicação com milhares de dispositivos simultaneamente.\newline{}
Tarefas realizadas:
\begin{itemize}
    \item Novas funcionalidades no sistema web;
    \item Correções e melhorias no MVP para torná-lo um produto mais robusto;
    \item Soluções para integração contínua;
    \item Comunicação real-time com os hardwares instalados nos clientes usando Nodejs;
    \item Criação de landing pages para validação de ideias com HTML5 e CSS3;
    \item Integração das landing pages com as ferramenta Google Analytics e Google Ads;
    \item Frameworks e tecnologias utilizadas:
      \begin{itemize}
          \item Ruby on Rails;
          \item Javascript;
          \item Scrum;
          \item Bootstrap;
          \item Nodejs;
          \item SemaphoreCI;
          \item MySQL;
          \item Git;
          \item Google Analytics e Google Ads;
      \end{itemize}
\end{itemize}}

\cventry{Julho 2015--Dezembro 2015}{Co-fundador}{}{Coremaster}{}{Co-fundador da empresa focada em desenvlver sites institucionais para pequenas empresas e estruturação e segurança das redes de computadores. Saí da empresa devido a divergências sobre a visão de negócio com o fundador.\newline{}
Tarefas realizadas:
\begin{itemize}
    \item Construção sites em HTML5 e CSS3;
    \item Planejamento das estratégias de marketing digital da empresa;
    \item Gestão do fluxo de caixa da empresa;
    \item Frameworks e tecnologias utilizadas:
      \begin{itemize}
          \item HTML5 and CSS3;
          \item Javascript;
          \item Jquery;
          \item Bootstrap;
          \item Wordpress;
      \end{itemize}
\end{itemize}}

\cventry{Setembro 2013--Julho 2014}{Estágio}{}{Banco do Brasil}{}{Dei Manutenção e desenvolvi novas funcionalidades no sistema web em Java usados pelas agências fora do país. Refatorei parte do sistema, aonde foi realizada modularização do sistema, limpeza do código, correção de bugs e testes do sistema. Foi iniciado o desenvolvimento de uma API RESTFull Java para acesso pelas filiais à dados da base da sede nacional.\newline{}
Tarefas realizadas:
\begin{itemize}
    \item Desenvolvimento de novas funcionalidades;
    \item Refatoração do sistema;
    \item Limpeza do código de bad smells e implementação de boas práticas;
    \item Manutenção na base de dados em PostgreSQL;
    \item Iniciado o desenvolvimento de uma API RESTFull Java para acesso pelas filiais à dados da base da sede nacional;
    \item Frameworks e tecnologias utilizadas:
      \begin{itemize}
          \item Java;
          \item Tomcat;
          \item JSF;
          \item framework FreeMarker;
          \item PostgreSQL;
          \item SVN;
      \end{itemize}
\end{itemize}}
