\section{Experiências}
\cventry{Setembro 2013--Julho 2014}{Estágio}{}{Banco do Brasil}{}{Manutenção e evolução de sistemas web em Java usados pelas agências fora do país. Refatorei parte do sistema, aonde foi realizada modularização do sistema, limpeza do código, correção de bugs e testes do sistema. Foi iniciado o desenvolvimento de uma API RESTFull Java para acesso pelas filiais à dados da base da sede nacional.\newline{}
Tarefas realizadas:
\begin{itemize}
    \item Refatoração e evolução de sistema Java web;
      \begin{itemize}
          \item Modularização do sistema;
          \item Limpeza do código de bad smells;
          \item Desenvolvimento de novas funcionalidades;
      \end{itemize}
    \item Manutenção na base de dados em PostgreSQL;
    \item Iniciado o desenvolvimento de uma API RESTFull Java para acesso pelas filiais à dados da base da sede nacional;
    \item Frameworks e tecnologias utilizadas:
      \begin{itemize}
          \item JSF;
          \item framework FreeMarker;
          \item PostgreSQL;
          \item SVN;
      \end{itemize}
    \item Contato: Carlos (61)98477-2967.
\end{itemize}}

\cventry{Julho 2015--Dezembro 2015}{Co-fundador}{}{Coremaster}{}{Co-fundador da empresa juntamente a um amigo, na categoria Simples Nacional. Empresa atuava no desenvolvimento de sites institucionais e estruturação e segurança de redes de computadores. Saí da empresa devido a divergências sobre a visão de negócio com o fundador.\newline{}
Tarefas realizadas:
\begin{itemize}
    \item Construção sites em HTML5 e CSS3;
    \item Planejamento das estratégias de marketing digital da empresa;
    \item Gestão do fluxo de caixa da empresa;
    \item Frameworks e tecnologias utilizadas:
      \begin{itemize}
          \item Jquery;
          \item Bootstrap;
          \item Wordpress;
      \end{itemize}
    \item Contato: Rafael Cançado (61)98102-8023.
\end{itemize}}

\cventry{Abril 2016--Dezembro 2016}{Desenvolvedor Ruby on Rails e Nodejs}{}{Preditiva}{}{Desenvolvedor web na startup, corrigindo bugs e implementando novas funcionalidades. Além do sistema da empresa, implantei soluções de deploy contínuo e construí diversas landing pages para validação de ideias pela equipe de marketing. Construí ainda um middleware para facilitar a comunicação em tempo real com os hardwares instalados no cliente e permitir a comunicação com milhares de dispositivos simultaneamente.\newline{}
Tarefas realizadas:
\begin{itemize}
    \item Novas funcionalidades no sistema web;
    \item Correções e melhorias no MVP para torná-lo um produto mais robusto;
    \item Soluções para integração contínua;
    \item Comunicação real-time com os hardwares instalados nos clientes usando Nodejs;
    \item Criação de diversas landing pages para validação de ideias com HTML5 e CSS3;
    \item Integração das landing pages com as ferramenta Google Analytics e Google Ads;
    \item Frameworks e tecnologias utilizadas:
      \begin{itemize}
          \item Ruby on Rails;
          \item Scrum;
          \item JQuery;
          \item Bootstrap;
          \item Nodejs;
          \item SemaphoreCI;
          \item MySQL;
          \item Git;
          \item Google Analytics e Google Ads;
      \end{itemize}
    \item Contato: Luiz (61)99203-3355.
\end{itemize}}

\cventry{Dezembro 2016--Abril 2019}{Desenvolvedor FullStack}{}{AIS Digital}{}{Desenvolvendo e dando manutenção em sistemas Ruby on Rails para grandes clientes, entre eles Banco do Brasil e Altran. Participação em projetos internacionais, desde a comunicação direta com o cliente, levantamento das necessidades com o mesmo e resolução deles. Atuei em equipes multi-disciplinares e dinâmicas, desenvolvendo o frontend ao lado dos designers e contruindo APIs para o frontend e os apps mobile acessarem. Palestrei e realizei workshop para a equipe de desenvolvimento da empresa sobre testes automatizados. \newline{}
Tarefas realizadas:
\begin{itemize}
    \item Desenvolvimento de sistemas do início e manutenção em sistemas legados em Ruby on Rails;
    \item Realizar palestra e workshop sobre testes automatizados;
    \item Estimar e projetar esforço e impacto de novas implementações nos projetos;
    \item Implementação do Scrum e Kanban nas equipes as quais participei;
    \item Atuação em equipes multi-disciplinares em formato de squads;
    \item Desenvolvimento de APIs para comunicação com frontend e aplicativos mobile;
    \item Implantação de soluções para integração e deploy contínuo, análise de código e monitoramento de falhas;
    \item Comunicação direta com cliente: atendendo a chamadas, sendo alocado no mesmo e realizando levantamento de necessidades e bugs com cliente nacional e internacional;
    \item Documentação do projeto, API e processos;
    \item Frameworks e tecnologias utilizadas:
      \begin{itemize}
          \item Ruby on Rails;
          \item Scrum e Kanban;
          \item JQuery;
          \item Backbone JS;
          \item npm e Vuejs;
          \item RSpec, Capistrano e Jest;
          \item MySQL e PostgreSQL;
          \item Docker;
          \item Git e SVN;
          \item Pipeline;
          \item New Relic e Sentry;
          \item SonarQube e Rubocop;
          \item Swagger, Wiki e Confluence;
          \item AWS e Jelastic;
      \end{itemize}
\end{itemize}}

\cventry{Abril 2019--Novembro 2019}{Desenvolvedor FullStack}{}{Ownee}{}{Desenvolvimento e melhoria da aplicação web (Ruby on Rails e Vuejs). Construindo novas funcionalidades, otimizando as requisições e as buscas no banco de dados, comunicando com APIs externas e criando componentes novos tanto no backend como no frontend. Foi feito uma refatoração completa, alterando o núcleo do sistema, a arquitetura e suas regras de negócio, aonde atuei ativamente na modelagem da nova versão. Atuei como analista de testes, sendo o responsável por escrevê-los e atualizá-los a cada mudança nos requisitos. Atuei como líder técnico ao lado do CTO e guiei os estagiários em suas atividades. \newline{}
Tarefas realizadas:
\begin{itemize}
    \item Modelagem e implementação de uma nova arquitetura a partir das mudanças no sistema e da versão antiga;
    \item Desenvolvimento de componentes e funcionalidades em Ruby on Rails e Vuejs;
    \item Acompanhamento e mentoria dos estagiários;
    \item Otimização de pesquisas no banco de dados;
    \item Refatorações no sistema para torná-lo mais rápido e escalável;
    \item Investigação e correção de bugs reportados pelo usuários;
    \item Comunicação entre frontend e backend e com APIs externas;
    \item Manutenção dos testes e criação dos novos;
    \item Frameworks e tecnologias utilizadas:
      \begin{itemize}
          \item Ruby on Rails e RSpec;
          \item Scrum;
          \item Javascript, Yarn e Vuejs;
          \item PostgreSQL;
          \item Docker;
          \item Git;
          \item New Relic;
          \item Wiki e AWS;
      \end{itemize}
\end{itemize}}

\cventry{Julho 2020--Março 2022}{Desenvolvedor FullStack}{}{Epy}{}{Desenvolvimento de soluções em Ruby on Rails, React e Javascript na AWS. Atuo em todas as fases do desenvolvimento, desde a concepção da nova funcionalidade, arquitetura e implementação. Monitoramento e otimização do sistema de forma constante. Realizei integrações com diversos sistemas externos. Atuei migrando o servidor Rails para uma arquitetura totalmente nova em microsserviços. Também fui responsável por ser mentor dos trainees da startup e ajudá-los nos seus desenvolvimentos. \newline{}
Tarefas realizadas:
\begin{itemize}
    \item Desenvolvimento de componentes e funcionalidades em Ruby on Rails e React;
    \item Implementação do sistema PWA no celular e comunicação em tempo real entre todas as partes (celular, servidor e desktop);
    \item Acompanhamento e mentoria dos trainees;
    \item Otimização de pesquisas no banco de dados;
    \item Refatorações no sistema para torná-lo mais rápido;
    \item Autenticação e integração com APIs externas;
    \item Modelagem e implementação de uma nova arquitetura em microsserviços usando lambdas da AWS;
    \item Frameworks e tecnologias utilizadas:
      \begin{itemize}
          \item Ruby on Rails;
          \item Javascript e Typescript;
          \item Yarn e React;
          \item PWA;
          \item Web sockets;
          \item AWS;
          \item Microsserviços;
          \item Scrum;
          \item PostgreSQL;
          \item Docker;
          \item Git;
          \item Sentry;
      \end{itemize}
\end{itemize}}
