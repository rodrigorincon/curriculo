\section{Presentations}

\begin{itemize}
    \item I did a workshop about algorithms and how to think solutions as a programming developer at Latin American Open Source Software Instalation Fair (FLISoL 2023), the biggest event for spreading open source in latin America, and at Campus Party 2023, focused on elementary school children;
    \item I did a workshop about the first steps with Javascript at Campus Party 2022, teaching basic concepts about programming using Javascript for teenagers and adults;
    \item I did some workshops and speeches in Epy company for the trainees during 2021 and 2022, teaching about good pratices, clean code, API integrations and specific components of the system;
    \item I did a workshop and speech about automated tests with RSpec in AIS Digital company in 2018, talking about what and how to test, the types of tests and how to write a test with RSpec in a Ruby on Rails project;
    \item I taught about basic concepts of programming for high school students through robotics in the Edubot project at University of Brasília in 2016, using an educational kit made from Arduino;
    \item I presented at the Science and Technology Meeting in University of Brasília (ECT 2011) a project of food re-education software for Android system with a prototype demonstration;
    \item I did a speech and had an article accepted at the Science and Technology Meeting in University of Brasília (ECT 2011) regarding a project of continued training and appropriate technologies for digital inclusion of disabled and elderly people; 
\end{itemize}


\section{Projects}

\begin{itemize}
    \item I participated in the Labkids organization during 2022 and 2023, teaching basic concepts about programming for children through robotics, using an educational kit made from Arduino;
    \item I participated in the Edubot project at University of Brasília in 2016, teaching basic concepts about programming to high school students through robotics, using an educational kit made from Arduino;
    \item I participated in the development of the Adobe Reader plugin for digital signatures validation in PAdES standards of ICP-Brasil at University of Brasília from 2015 to 2016, using OpenSSL and C++ language;
    \item I participated in the project "Solidarity Carpool" at the University of Brasília from 2013 to 2014, as Java and PHP programmer and responsible for the modeling and implementation of the database;
    \item I participated in a development project in University of Brasília in 2012, in the database and Java development area;
    \item I participated in the research project Casa Brasil of Adapte (Association for Supporting People with Disabilities) by University of Brasília in 2011, in the area of digital inclusion of disabled and elderly people;
\end{itemize}