\section{Professional Experiences}
\cventry{September 2013--July 2014}{Internship}{}{Bank of Brazil}{}{Maintenance and evolution of Java web systems used by agencies outside the country. Refactoring part of the system, where I modularized the system, cleaned the code, fix bugs and tested funcionalities. Was started the development of a Java RESTFull API for access data by subsidiary companies. \newline{}
Tasks performed:
\begin{itemize}
    \item Refactoring and evolution of Java web systems;
      \begin{itemize}
          \item System modularization;
          \item Bad smells code cleanup;
          \item Development of new funcionalities;
      \end{itemize}
    \item PostgreSQL database maintenance;
    \item Started the development of a Java RESTFull API for access to data from the national headquarters database;
    \item Frameworks and technologies used:
      \begin{itemize}
          \item JSF;
          \item framework FreeMarker;
          \item PostgreSQL;
          \item SVN;
      \end{itemize}
    \item Contact: Carlos +55 6198477-2967.
\end{itemize}}

\cventry{July 2015--December 2015}{Co-founder}{}{Coremaster}{}{Co-founder of a small-scale company focused on the development of institutional websites and the structuring and security of computer networks. I left the company due to disagreements about the business vision with the founder.\newline{}
Tasks performed:
\begin{itemize}
    \item Website development in HTML5 and CSS3;
    \item Planning of the company's digital marketing strategies;
    \item Management of the company's cash flow;
    \item Frameworks and technologies used:
      \begin{itemize}
          \item Jquery;
          \item Bootstrap;
          \item Wordpress;
      \end{itemize}
    \item Contact: Rafael Cançado +55 6198102-8023.
\end{itemize}}

\cventry{April 2016--December 2016}{Ruby on Rails e Nodejs developer}{}{Preditiva}{}{Web developer in the startup, fixing bugs and developing new features. I implanted solutions for continuous delivery and built many landing pages for idea validations to the marketing team. Built a middleware to facilitate communication in real time with the hardwares installed in the client and allow communication with thousands of devices simultaneously.\newline{}
Tasks performed:
\begin{itemize}
    \item New features in the web system;
    \item Fixes and improvements in the MVP;
    \item Solutions for continuous delivery;
    \item Real-time comunication with the hardwares installed in the clients using Nodejs;
    \item Development of landing pages to validate ideas with HTML5 and CSS3;
    \item Integration of landing pages with Google Analytics and Google Ads;
    \item Frameworks and technologies used:
      \begin{itemize}
          \item Ruby on Rails;
          \item Scrum;
          \item JQuery;
          \item Bootstrap;
          \item Nodejs;
          \item SemaphoreCI;
          \item MySQL;
          \item Git;
          \item Google Analytics and Google Ads;
      \end{itemize}
    \item Contact: Luiz +55 6199203-3355.
\end{itemize}}

\cventry{December 2016--April 2019}{FullStack developer}{}{AIS Digital}{}{Developing and maintaining Ruby on Rails systems for large clients, such as Bank of Brazil and Altran. Participation in international projects, since the direct communication with the client, surveying its needs and providing their resolution. Worked in dynamic and multidisciplinary teams, developing the frontend alongside designers and building APIs to frontend and mobile apps. I lectured and did a workshop for the company's developers about automated tests.\newline{}
Tasks performed:
\begin{itemize}
    \item System development from the beginning and maintenance of legacy systems in Ruby on Rails;
    \item Lecture and do workshop about automated tests;
    \item Estimate and project the effort and the impact of new features in the projects;
    \item Implementation of Scrum and Kanban in the teams in which I participate; 
    \item Act in multidisciplinary teams in Squad format;
    \item API development for communication with frontend and mobile apps;
    \item Implementation of CI and CD solutions, code analysis and breaks monitoring;
    \item Direct communication with the client: answering calls, being allocated inside the clients and making survey of needs and bugs with the national and international clients;
    \item Project documentation, API and processes;
    \item Frameworks and technologies used:
      \begin{itemize}
          \item Ruby on Rails;
          \item Scrum and Kanban;
          \item JQuery;
          \item Backbone JS;
          \item npm and Vuejs;
          \item RSpec, Capistrano and Jest;
          \item MySQL and PostgreSQL;
          \item Docker;
          \item Git and SVN;
          \item Pipeline;
          \item New Relic and Sentry;
          \item SonarQube and Rubocop;
          \item Swagger, Wiki and Confluence;
          \item AWS and Jelastic;
      \end{itemize}
\end{itemize}}

\cventry{April 2019--November 2019}{FullStack Developer}{}{Ownee}{}{Developing and improving the web application (Ruby on Rails and Vuejs). Building new features, optimizing requests and database queries, communicating with external APIs and creating new components in backend and frontend. A complete refactoring was done, changing the core of system, the achitecture and its business rules, where I actively worked on modeling the new version. I acted as test analyst, being responsible to write and update them at each change in the requirements. I acted as tech leader next to the CTO and guided the interns in your activities. \newline{}
Tarefas realizadas:
\begin{itemize}
    \item Modeling and implementing a new architecture from system changes and the old version
    \item Development of components and features in Ruby on Rails and Vuejs;
    \item Trainees mentoring and supervision;
    \item Database queries optimization;
    \item System refactor to make it faster and scalable;
    \item Investigation and bug fixes reported by the users;
    \item Communication between frontend and backend and with external APIs;
    \item Maintenance and creation of tests;
    \item Frameworks and technologies used:
      \begin{itemize}
          \item Ruby on Rails and RSpec;
          \item Scrum;
          \item Javascript, Yarn and Vuejs;
          \item PostgreSQL;
          \item Docker;
          \item Git;
          \item New Relic;
          \item Wiki and AWS;
      \end{itemize}
\end{itemize}}

\cventry{July 2020--Currently}{FullStack Developer}{}{Epy}{}{Developing solutions with Ruby on Rails, React and Javascript in AWS. I'm working in all development phases, since the conception of new feature, design and development. Monitoring and optimizing the system constantly. I did integration with many external systems. I'm migrating the Rails server to a completely new microsservice architecture. I was also responsible for mentoring the startup trainees and helping them in their development. \newline{}
Tarefas realizadas:
\begin{itemize}
    \item Components and functionalities development in Ruby on Rails e React;
    \item Development of PWA system on mobile and real time communication between all parts (mobile, server and desktop);
    \item Trainee monitoring and mentoring;
    \item Database queries optimizing;
    \item Refactoring to make it faster;
    \item Autentication and integration with external APIs;
    \item Modeling e implementation of new microsservice architecture using AWS lambdas;
    \item Frameworks and technologies used:
      \begin{itemize}
          \item Ruby on Rails;
          \item Javascript and Typescript;
          \item Yarn and React;
          \item PWA;
          \item Web sockets;
          \item AWS;
          \item Microsservices;
          \item Scrum;
          \item PostgreSQL;
          \item Docker;
          \item Git;
          \item Sentry;
      \end{itemize}
\end{itemize}}
